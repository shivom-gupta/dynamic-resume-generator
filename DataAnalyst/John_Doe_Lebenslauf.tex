% resume_template.tex
\documentclass[9pt,a4paper,ragged2e,withhyper]{altacv}

\geometry{left=0.8cm,right=0.8cm,top=0.8cm,bottom=0.8cm,columnsep=0.75cm}
\usepackage{paracol}
\usepackage[none]{hyphenat}
\usepackage{microtype}


\ifxetexorluatex
  \setmainfont{Roboto Slab}
  \setsansfont{Lato}
  \renewcommand{\familydefault}{\sfdefault}
\else
  \usepackage[rm]{roboto}
  \usepackage[defaultsans]{lato}
  \renewcommand{\familydefault}{\sfdefault}
\fi

\definecolor{PrimaryColor}{HTML}{2E2E2E}
\definecolor{SecondaryColor}{HTML}{0039AC}
\definecolor{ThirdColor}{HTML}{FFAA00}
\definecolor{BodyColor}{HTML}{666666}
\definecolor{EmphasisColor}{HTML}{2E2E2E}
\definecolor{BackgroundColor}{HTML}{E2E2E2}

\colorlet{name}{PrimaryColor}
\colorlet{tagline}{SecondaryColor}
\colorlet{heading}{PrimaryColor}
\colorlet{headingrule}{ThirdColor}
\colorlet{subheading}{SecondaryColor}
\colorlet{accent}{SecondaryColor}
\colorlet{emphasis}{EmphasisColor}
\colorlet{body}{BodyColor}
\pagecolor{BackgroundColor}

\renewcommand{\namefont}{\Huge\rmfamily\bfseries}
\renewcommand{\personalinfofont}{\small\bfseries}
\renewcommand{\cvsectionfont}{\Large\rmfamily\bfseries}
\renewcommand{\cvsubsectionfont}{\large\bfseries}

\renewcommand{\itemmarker}{{\small\textbullet}}
\renewcommand{\ratingmarker}{\faCircle}

\newenvironment{sloppypar*}{\sloppy\ignorespaces}{\par}

\begin{document}
    \sloppy
    \name{ John Doe }
    \tagline{ Datenanalyst }
    \photoL{4cm}{../me.png}

    \personalinfo{
        \email{ john.doe.email@email.com }\smallskip
        \phone{ +49 123 4567890 }
        \location{ Berlin, Germany }\\
        \linkedin{ john-doe }
        \github{ john-doe }
        \homepage{ https://john-doe.github.io/ }
    }
    
    \makecvheader
    \columnratio{0.25}
    \begin{paracol}{2}
        
        \cvsection{ Technische \\ Kenntnisse }
            \begin{sloppypar*}
                \cvtags{
                    \noindent Machine Learning/true, Data Science, Data Analysis/true, Data Visualization, Statistical Analysis/true, Data Preprocessing/true, CI CD, EDA, Hypothesis Testing                }
                \cvtags{
                    \noindent Python/true, SQL/true, Excel, Tableau/true, PowerBI/true, Git/true, AWS/true, Linux/true, R                }
            \end{sloppypar*}

        \cvsection{ Soziale \\ Kompetenzen }
            \begin{sloppypar*}
                \cvtags{
Problemlösungskompetenz/true, Analytisches Denken/true, Teamarbeit/true, Kommunikation/true                }
            \end{sloppypar*}
        
        \cvsection{ Aktuell in \\ Weiterbildung }
            \begin{sloppypar*}
                \cvtags{
Deutsche Sprache/true                }
            \end{sloppypar*}
        
        \cvsection{ Sprachkenntnisse }
            \cvlang{ Englisch }{ Fließend }\\
            \cvlang{ Deutsch }{ Selbständig }\\
            \cvlang{ Spanisch }{ Fließend }\\
            \cvlang{ Französisch }{ Muttersprache }\\
        
        \switchcolumn
        
        \cvsection{ Profil }
        Zuverlässiger Datenanalyst mit Zertifizierungen in SQL, Python, Tableau und PowerBI. Erfahrung in der Erstellung interaktiver Dashboards, statistischer Analysen und der Verbesserung der Datenqualität für Geschäftskunden. Kunden gaben durchweg positives Feedback zu meinen freiberuflichen Projekten.
        
        \cvsection{ Berufserfahrung }
            \cvevent{ Machine Learning Forscher }{ AI Research Lab GmbH }{ August 2023 - Heute }{ Berlin, Germany }
            \begin{itemize}
                \item Entwicklung und Optimierung fortschrittlicher Deep-Learning-Modelle zur Verbesserung der Vorhersagegenauigkeit von wissenschaftlichen Daten.
                \item Systematische Bewertung der Modellleistung und Generalisierung über verschiedene Datensätze und Eingabegrößen hinweg.
                \item Analyse des Einflusses von Datenrepräsentationen und Modellarchitektur auf die Robustheit der Vorhersagen.
                \item Mitarbeit an einer wissenschaftlichen Veröffentlichung für eine Fachzeitschrift.
            \end{itemize}
            \vspace{0.5em}
            \cvevent{ Machine Learning Ingenieur }{ Tech Innovations AG }{ Januar 2023 - Juli 2023 }{ Berlin, Germany }
            \begin{itemize}
                \item Entwicklung und Implementierung von maschinellen Lernmodellen für die Echtzeit-Datenverarbeitung und -analyse.
                \item Entwicklung einer Computer Vision-Anwendung zur Klassifizierung von Bildern mit Deep Learning-Modellen, die eine hohe Genauigkeit auf Testdatensätzen erreicht.
                \item Zusammenarbeit mit funktionsübergreifenden Teams zur Integration von KI-Lösungen in bestehende Systeme zur Effizienzsteigerung.
                \item Mitwirkung an der Entwicklung eines KI-Agenten für ein Wortspiel mithilfe von Deep Q-Networks in PyTorch.
            \end{itemize}
            \vspace{0.5em}
        
        \cvsection{ Akademische Ausbildung }
            \cvevent{ B.Sc. Computational Science }{ University of Berlin }{ Oktober 2020 - September 2024 }{ Berlin, Germany }
            \begin{itemize}
                \item Note: 1,9 (gut)
                \item Abschlussarbeit: "Analyse komplexer Systeme mit maschinellem Lernen"
                \item Relevante Kurse: Fortgeschrittene Algorithmen, Maschinelles Lernen, Big-Data-Technologien, Numerische Methoden, Software Engineering, Lineare Algebra und Analysis.
            \end{itemize}
            \vspace{0.5em}
            \cvevent{ Zertifizierungskurs zum Datenanalysten }{ Online Learning Academy }{ Februar 2022 - August 2022 }{ Remote }
            \begin{itemize}
                \item Absolvierung eines umfassenden Programms zur Datenanalyse mit Fokus auf SQL, Python, Tableau, PowerBI und Statistik.
                \item Praktische Erfahrung in Datenbereinigung, Visualisierung und Erstellung von Dashboards.
            \end{itemize}
            \vspace{0.5em}
        
        \cvsection{ Ausgewählte Projekte }
        \cvevent
            { Groß angelegte Textanalyse }
            { \cvreference{\faGithub}{ https://github.com/john-doe/large-scale-text-analysis } }
            {}
            {}
        Mitarbeit beim Aufbau eines GPU-Clusters zur Extraktion spezifischer Aussagen aus einem großen Korpus von Webarchiv-Dateien mit Regex und ML-Algorithmen, gefolgt von einer NLP-Analyse der Daten.\\
        \vspace{0.5em}
        \cvevent
            { Umweltdatenanalyse }
            { \cvreference{\faGithub}{ https://github.com/john-doe/environmental-data-analysis } }
            {}
            {}
        Durchführung einer SQL-Analyse zur Bewertung von Veränderungen der CO2-Emissionen in verschiedenen Ländern mit konkreten Handlungsempfehlungen zur Adressierung von Umwelttrends.\\
        \vspace{0.5em}
        \cvevent
            { Modell zur Finanzrisikobewertung }
            { \cvreference{\faGithub}{ https://github.com/john-doe/financial-risk-assessment } }
            {}
            {}
        Entwicklung eines maschinellen Lernmodells zur Vorhersage von Finanzrisiken mit hoher Genauigkeit auf Testdaten. Dabei wurden Kenntnisse in Datenvorverarbeitung, Feature Engineering und Modelloptimierung demonstriert.\\
        \vspace{0.5em}

        
    \end{paracol}

\end{document}